\documentclass[a4paper]{article}
% \documentclass[10pt]{scrartcl}

% ------------------------------------------------ %
%                                                  %
%              Fonts and typography                %
%                                                  %
% ------------------------------------------------ %

%% Typography
\usepackage[no-math]{fontspec}
\usepackage{color}

% Latin Modern Typewriter
% \usepackage{lmodern}
% \renewcommand*\familydefault{\ttdefault} %% Only if the base font of the document is to be typewriter style
% \usepackage[T1]{fontenc}
%
% % Open Sans
% \usepackage[default,osfigures,scale=0.95]{opensans} %% Alternatively
% %% use the option 'defaultsans' instead of 'default' to replace the
% %% sans serif font only.
% \usepackage[T1]{fontenc}

% \usepackage{opensans}

%% Fonts
\setmainfont{Roboto}
\setsansfont{Roboto}
\setmonofont{Roboto}

%% Sans serif as main font
\renewcommand{\familydefault}{\sfdefault}

%% Set Sans font in headings
\usepackage{sectsty}
\allsectionsfont{\sffamily}

%% Set polyglossia language
\usepackage{polyglossia}
\setdefaultlanguage{english}
% disable hyphenation
%\hyphenpenalty = 10000

%% Emphasizing inline code output
\usepackage{alltt}


% ------------------------------------------------ %
%                                                  %
%              Settings for knitr                  %
%                                                  %
% ------------------------------------------------ %

%% for Shaded environment in knitr
\usepackage{framed}
\usepackage{fancyhdr}

%% landscape mode
\usepackage{lscape}

%% Encoding
%\usepackage[UTF-8]{inputenc}

% rowspacing
\usepackage{setspace}

% ------------------------------------------------ %
%                                                  %
%              Page settings                       %
%                                                  %
% ------------------------------------------------ %

% \input{preamb_a4.tex}

%% Use full page in book style
\usepackage[top=3.5cm, bottom=3.5cm, left=1.5cm, right=1.5cm]{geometry}

%% Set line spacing
\usepackage{setspace}
\setstretch{1.0}

%% Disable paragraph indentation
\usepackage{parskip}

%% Custom headers
\usepackage{fancyhdr}

%% Multicolumns
\usepackage{multicol}

% ------------------------------------------------ %
%                                                  %
%                  Headings                        %
%                                                  %
% ------------------------------------------------ %

\usepackage{titlesec, blindtext, color}

% spacin before and after the section titles
\titlespacing*{\section}{0pt}{2.5ex}{0.5ex}
\titlespacing*{\subsection}{0pt}{0.2ex}{0.2ex}

%\definecolor{SkyBlue}{HTML}{3465A4}
\definecolor{green}{RGB}{26,150,65}
\definecolor{FAOBlue}{RGB}{80,135,206}

\newcommand{\hsp}{\hspace{20pt}}
% add the green vertical line after 1st level heading
\titleformat{\section}[hang]{\large\bfseries}{\thesection\hsp\textcolor{FAOBlue}{|}\hsp}{0pt}{\large\bfseries}

\usepackage{enumerate}

%% Disable section numbers
% \setcounter{secnumdepth}{1}

% ------------------------------------------------ %
%                                                  %
%               Table of content                   %
%                                                  %
% ------------------------------------------------ %

% dotted lines for table of content

\makeatletter
\renewcommand*\l@section{\@dottedtocline{1}{1.5em}{2.3em}}
\makeatother
\setcounter{tocdepth}{1}

% ------------------------------------------------ %
%                                                  %
%               % Colors                           %
%                                                  %
% ------------------------------------------------ %
\usepackage{xcolor}

%% Tango color scheme
\definecolor{SkyBlue}{HTML}{3465A4}
\definecolor{DarkSkyBlue}{HTML}{204A87}
\definecolor{Plum}{HTML}{75507B}
\definecolor{ScarletRed}{HTML}{CC0000}
\definecolor{Aluminium1}{HTML}{EEEEEC}
\definecolor{Aluminium6}{HTML}{2e3436}
\definecolor{Black}{HTML}{000000}

% ------------------------------------------------ %
%                                                  %
%               Links                              %
%                                                  %
% ------------------------------------------------ %

%% Hyperref
\usepackage[colorlinks, breaklinks, bookmarks]{hyperref}

\hypersetup {
  linkcolor = Aluminium6,
  citecolor = DarkSkyBlue,
  filecolor = DarkSkyBlue,
  urlcolor = DarkSkyBlue
}

%% Don’t use Mono font for URLs
\urlstyle{same}


% ------------------------------------------------ %
%                                                  %
%               Images & tables                    %
%                                                  %
% ------------------------------------------------ %

% Images
\usepackage{graphicx}
% tables
\usepackage{longtable}
\usepackage{booktabs}
\usepackage{float}
\restylefloat{table}
\usepackage{wrapfig}

 \renewcommand\arraystretch{1.1} % table rowheight
% see: http://everythingyouforgetaboutlatex.blogspot.fi/2008/10/formatting-tables.html
\setlength{\intextsep}{-0.1cm} % spacing between floats


% SHADOWS FOR FLAGS

\usepackage{tikz}
\usetikzlibrary{shadows,calc}

% code adapted from https://tex.stackexchange.com/a/11483/3954

% some parameters for customization
\def\shadowshift{3pt,-3pt}
\def\shadowradius{6pt}

\colorlet{innercolor}{black!60}
\colorlet{outercolor}{gray!05}

% this draws a shadow under a rectangle node
\newcommand\drawshadow[1]{
    \begin{pgfonlayer}{shadow}
        \shade[outercolor,inner color=innercolor,outer color=outercolor] ($(#1.south west)+(\shadowshift)+(\shadowradius/2,\shadowradius/2)$) circle (\shadowradius);
        \shade[outercolor,inner color=innercolor,outer color=outercolor] ($(#1.north west)+(\shadowshift)+(\shadowradius/2,-\shadowradius/2)$) circle (\shadowradius);
        \shade[outercolor,inner color=innercolor,outer color=outercolor] ($(#1.south east)+(\shadowshift)+(-\shadowradius/2,\shadowradius/2)$) circle (\shadowradius);
        \shade[outercolor,inner color=innercolor,outer color=outercolor] ($(#1.north east)+(\shadowshift)+(-\shadowradius/2,-\shadowradius/2)$) circle (\shadowradius);
        \shade[top color=innercolor,bottom color=outercolor] ($(#1.south west)+(\shadowshift)+(\shadowradius/2,-\shadowradius/2)$) rectangle ($(#1.south east)+(\shadowshift)+(-\shadowradius/2,\shadowradius/2)$);
        \shade[left color=innercolor,right color=outercolor] ($(#1.south east)+(\shadowshift)+(-\shadowradius/2,\shadowradius/2)$) rectangle ($(#1.north east)+(\shadowshift)+(\shadowradius/2,-\shadowradius/2)$);
        \shade[bottom color=innercolor,top color=outercolor] ($(#1.north west)+(\shadowshift)+(\shadowradius/2,-\shadowradius/2)$) rectangle ($(#1.north east)+(\shadowshift)+(-\shadowradius/2,\shadowradius/2)$);
        \shade[outercolor,right color=innercolor,left color=outercolor] ($(#1.south west)+(\shadowshift)+(-\shadowradius/2,\shadowradius/2)$) rectangle ($(#1.north west)+(\shadowshift)+(\shadowradius/2,-\shadowradius/2)$);
        \filldraw ($(#1.south west)+(\shadowshift)+(\shadowradius/2,\shadowradius/2)$) rectangle ($(#1.north east)+(\shadowshift)-(\shadowradius/2,\shadowradius/2)$);
    \end{pgfonlayer}
}

% create a shadow layer, so that we don't need to worry about overdrawing other things
\pgfdeclarelayer{shadow} 
\pgfsetlayers{shadow,main}

\newsavebox\mybox
\newlength\mylen

\newcommand\shadowimage[2][]{%
\setbox0=\hbox{\includegraphics[#1]{#2}}
\setlength\mylen{\wd0}
\ifnum\mylen<\ht0
\setlength\mylen{\ht0}
\fi
\divide \mylen by 120
\def\shadowshift{\mylen,-\mylen}
\def\shadowradius{\the\dimexpr\mylen+\mylen+\mylen\relax}
\begin{tikzpicture}
\node[anchor=south west,inner sep=0] (image) at (0,0) {\includegraphics[#1]{#2}};
\drawshadow{image}
\end{tikzpicture}}


% ------------------------------------------------ %
%                                                  %
%               First page graphics                %
%                                                  %
% ------------------------------------------------ %

%\usepackage[showframe,textwidth=3.0in]{geometry}
\usepackage{xstring}% for string comparrison
\usepackage{calc}%    for \widthof
\usepackage{pgf}%     for math calclations

% \newlength{\Size}
% \newcommand*{\PositionText}[3][l]{%
%     \IfStrEqCase{#1}{%
%         {l}{\noindent\hspace{#2}#3}%
%         {c}{\pgfmathsetlength{\Size}{#2-0.5*(\widthof{#3})}\noindent\hspace{\Size}#3}%
%         {r}{\pgfmathsetlength{\Size}{#2-1.0*(\widthof{#3})}\noindent\hspace{\Size}#3}%
%         }[\PackageError{PositionText}
%             {\MessageBreak Unrecognized alignment: #1.\MessageBreak
%             Valid alignments are are `l`, `c`, `r'}{}]%
% }%


% First page background graphics
% \usepackage{eso-pic}
% \newcommand\firstpagebackground{%
% \put(0,0){%
% \parbox[b][\paperheight]{\paperwidth}{%
% \vfill
% \centering
% \includegraphics[width=\paperwidth,height=\paperheight,%
% keepaspectratio]{bg_1st_page.pdf}%
% \vfill
% }}}
% %
% % Embedding
% \usepackage{pdfpages}

% % ------------------------------------------------ %
% %                                                  %
% %               Absolute text positions            %
% %                                                  %
% % ------------------------------------------------ %
%
%
% \usepackage[absolute]{textpos}
%
% \setlength{\TPHorizModule}{30mm}
% \setlength{\TPVertModule}{\TPHorizModule}
% \textblockorigin{10mm}{10mm} % start everything near the top-left corner
% \setlength{\parindent}{0pt}


% Modifying the title

\makeatletter
\renewcommand{\maketitle}{\bgroup\setlength{\parindent}{0pt}
\begin{flushleft}
  \LARGE \@title
  % \large \@author
  % \newline
  % \small \@date

  % \textsc{\normalsize \@daffiliation }

  % \hrulefill
\end{flushleft}\egroup
}
